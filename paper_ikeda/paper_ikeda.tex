\documentclass[12pt,eclepsfx,a4j,twoside,openright]{jreport}
\usepackage[dvipdfmx]{graphicx}
\usepackage{./tex_files/hangcaption}
%\usepackage{amssymb}
\usepackage{./tex_files/fancyheadings}
\usepackage{pifont}
\usepackage{url}
%\usepackage{amsmath}
\usepackage{./tex_files/cite}
\renewcommand\citeleft{}
\renewcommand\citeright{}
\renewcommand\citeform[1]{[#1]}
\usepackage{./tex_files/here}
\usepackage[subrefformat=parens]{subcaption}
\usepackage{listings, jlisting}
\lstset{
	language={Python},
	basicstyle={\small},
	identifierstyle={\small},
	commentstyle={\small\ttfamily},
	keywordstyle={\small\bfseries},
	ndkeywordstyle={\small},
	stringstyle={\small\ttfamily},
	frame={tb},
	breaklines=true,
	columns=[1]{fullflexible},
	numbers=left,
	xrightmargin=0zw,
	xleftmargin=0zw,
	numberstyle={\scriptsize},
	stepnumber=1,
	numbersep=1zw,
	morecomment=[l]{//}
}

\begin{document}
\clearpage
\thispagestyle{empty}
\vspace*{20pt}
%\newpage
\thispagestyle{empty}

%/**********************************************************************/
%		表紙
%/**********************************************************************/

\thispagestyle{empty}
\vspace*{5mm}
\begin{center}
{\LARGE 卒業論文 \\}	%卒業論文,修士論文,博士論文のいずれか
\vspace{15mm}
\baselineskip=13mm
{\Huge{LED照明を用いたデジタルサイネージに対する新しい情報付加手法の提案}}	%論文のタイトル,改行したい場合は\\で区切る
\vspace{7mm}

{\Large
\[ \mbox{Proposal of a new information addiction method}\choose
\mbox{for digital signage using LED lighting}\]}		%英語のタイトルをつけるとかっこいい

\vspace{45mm}
%{\LARGE 2006年\\}
{\Huge{池田 新}\\}		%あなたの名前



\vspace{10mm}
{\LARGE 立命館大学理工学部\\電子情報工学科\\		%所属です,学部と研究科は所属が違うので注意
}
\vspace{10mm}
{\LARGE 2023年12月\\}		%日付です
\end{center}

\newpage
\thispagestyle{empty}
 
 
\newpage
%/**********************************************************************/
%		概要
%/**********************************************************************/
\addcontentsline{toc}{chapter}{内容梗概}		%A4で1ページから2ページで良いです.

\pagenumbering{roman}
%\baselineskip=32pt
\begin{abstract}
\thispagestyle{plain}

\end{abstract}


%========================================
%========================================
%    論文本体
%========================================
%========================================
%\setcounter{tocdepth}{3}
\pagestyle{fancy}
\setcounter{page}{3}
%\pagenumbering{roman}
\tableofcontents
%\clearpage
\listoffigures
%\clearpage
\listoftables
\clearpage

\lhead{第 \thechapter 章} \rhead{}		%ここから各章毎のファイルを作成して論文とします.
\input{chptr1}
\input{chptr2}
\input{chptr3}
\input{chptr4}
\input{chptr5}
\input{chptr6}
\input{chptr7}


%------------------------------------------------------------------
% 参考文献
%------------------------------------------------------------------

\addcontentsline{toc}{chapter}{参考文献}%参考文献は大変重要です,bibtexというものを使用すると使い回しができて便利です.
\bibliographystyle{./tex_files/ieice_e}
\bibliography{./tex_files/e_fmcam_ref}		%これがbibtexファイルです.参考文献のリストです.

%------------------------------------------------------------------
%付録
%------------------------------------------------------------------

\appendix
\lhead{付録} \rhead{}
\input{huroku}
\newpage

%------------------------------------------------------------------
\chapter*{謝辞}							%謝辞は必ず書きます.好きにかけるのでどんどん書きましょう.
\lhead{謝辞} \rhead{}
\addcontentsline{toc}{chapter}{謝辞}
%-----------------------------------------------------------------
\vspace{15mm}
\begin{flushright}
2023年3月 立命 太郎
\end{flushright}
%------------------------------------------------------------------
%------------------------------------------------------------------
\chapter*{研究業績リスト}	
\addcontentsline{toc}{chapter}{発表論文リスト}
\lhead{発表論文リスト} \rhead{}
%------------------------------------------------------------------
%皆さんの研究業績をここにどんどん書いてきましょう!筆頭でも共著でも良いです.少しでも良いので書くとかっこいいです.修士の場合は論文と国際学会は欲しいですね.

\small
\noindent
{\bf 【国内研究会等発表】}		%ここに展示会参加等を書きましょう
\begin{itemize}
	\item
	\underline{立命太郎}, 逢坂京太郎, 安藤義男,竹 信孝, 熊木武志, "2nmプロセスルールのSoCの製造技術可能性, "
	ET\&IoT West 2021, Jul., 2021.
\end{itemize}
{\bf 【国外研究会等発表】}
\begin{itemize}
	\item
	\underline{Taro Ritsumei} and Takeshi Kumaki, "Possibility of 2nm process rule SoC manufacturing technology,"
	International Computers and communications (ICC), 2021.

\end{itemize}
{\bf 【その他研究活動】}		%ここに展示会参加等を書きましょう
\begin{itemize}
	\item 
	ET\&IoT West 2021, Jul., 2021.
	\item 
	ET\&IoT 2021, Nov., 2021.

\end{itemize}
{\bf 【受賞】}		%ここに展示会参加等を書きましょう
\begin{itemize}
	\item
	\underline{Yuta Moritake}, Yutaro Shimomura, Ryuya Kirihara, Yuki Hirota, Xiangbo Kong and Takeshi Kumaki, "Development of invisible information lighting display "Stego-panel IV","
	IEEE Grobal Conference on Consumer Electronics (GCCE), Excellent Demo! award, Gold prize, Oct., 2021.
\end{itemize}
\clearpage

\end{document}